\documentclass[UTF8]{article}
\usepackage{ctex}
\usepackage{anyfontsize}
\usepackage{graphicx} % Required for inserting images
\usepackage{amsmath}
\usepackage{physics}
\usepackage{tcolorbox}
\usepackage{ctex}
\usepackage{tikz}
\usepackage{bookmark}
\usepackage[a4paper, top=2cm, bottom=2cm,right=2cm, left=2cm]{geometry}
\usepackage{esint}
\usepackage{booktabs}
% \usepackage{newtxmath}


\usepackage{fancyhdr}
\pagestyle{fancy}
\fancyhf{} % 清空当前设置的页眉和页脚
\lhead{\leftmark} % 设置左边的页眉内容为当前子章节标题
\chead{} % 中间留空
\rhead{\thepage} % 设置右边的页眉内容为页码



\title{Some math problem}
\author{}
\date{}

\begin{document}
\maketitle
\thispagestyle{empty}
\newpage
\section{微积分}
\subsection{多元函数}
\noindent 下面给出$f(x,y)$在$(a,b)$处的泰勒展开:
\begin{align*}
    f(x,y) = & f(a,b)+f_x(a,b)(x-a)+f_y(a,b)(y-b)+ \\&\frac{1}{2!}(f_{xx}(a,b)(x-a)^2+f_{xy}(a,b)(x-a)(y-b)\\&+f_{yy}(a,b)(y-b)+\cdots)
\end{align*}
接下来是关于隐函数求导的内容,这个非常容易出错:
例如,给出如下方程:
\begin{align*}
    \begin{cases}
        x = r\cos \theta \\
        y=r \sin \theta
    \end{cases}
    \Rightarrow
    \theta = \arctan \frac{y}{x}
\end{align*}
分析$\frac{\partial x}{\partial \theta}$和$\frac{\partial \theta}{\partial x}$两者是否互为倒数,以前我的第一印象会告诉我这两个互为倒数,但是,根据偏导理论:
\begin{align*}
    \frac{\partial x}{\partial \theta}\mid_r=-y\quad\frac{\partial \theta}{\partial x}|_y =\frac{-y}{r^2}
\end{align*}
可以看到两者并不是倒数的关系,**因为求导时,视为常数的变量不一致**,要想使其成为导数,则将视为常数的变量变为一致,这里,改变第一个公式:
$$
    x = y\cot \theta \Rightarrow \frac{\partial x}{\partial \theta}|_y = \frac{-y}{\sin^2 \theta}
$$
可以看到:
$$\frac{\partial x}{\partial \theta }\mid_ y \cdot \frac{\partial \theta }{\partial x}\mid _y = 1 $$
在进行坐标变换时,
\begin{align*}
    \begin{cases}
        dx = \frac{\partial x}{\partial r}\mid _\theta  dr-\frac{\partial x}{\partial \theta }\mid _r   d\theta \\
        dy =\frac{\partial y}{\partial r}\mid _\theta  dr-\frac{\partial y}{\partial \theta }\mid _r   d\theta
    \end{cases}
    \Rightarrow
    \begin{bmatrix}
        dx \\dy
    \end{bmatrix}=A
    \begin{bmatrix}
        dr \\d\theta
    \end{bmatrix}
    =\begin{bmatrix}
         \frac{\partial x}{\partial r} \mid_\theta & \frac{\partial x}{\partial \theta }\mid _r \\
         \frac{\partial y}{\partial r} \mid_\theta & \frac{\partial y}{\partial \theta }\mid _r
     \end{bmatrix}
    \begin{bmatrix}
        dr \\d\theta
    \end{bmatrix}
\end{align*}
相应的坐标逆变换为:
\begin{align*}
    \begin{bmatrix}
        dr \\d\theta
    \end{bmatrix}=A^{-1}
    \begin{bmatrix}
        dx \\dy
    \end{bmatrix}
    =\begin{bmatrix}
         \frac{\partial r}{\partial x} \mid_y       & \frac{\partial r}{\partial y }\mid _x       \\
         \frac{\partial \theta }{\partial x} \mid_y & \frac{\partial \theta }{\partial y }\mid _x
     \end{bmatrix}
    \begin{bmatrix}
        dx \\dy
    \end{bmatrix}
\end{align*}
变换矩阵$A$和$A^{-1}$的对应矩阵元并不是互为倒数,但是$AA^{-1}=1$
\subsection{莱布尼兹求导法}
二项式展开公式(以9次为例,n次类似):$(a+b)^9=C_9^0 a^0b^9+C_9^1a^1b^8+\cdots+C_9^9 a^9b^0$,
由此受到启发,对于$(ab)^{(9)}=C_9^0 a^{(0)}b^{(9)}+C_9^1a^{(1)}b^{(8)}+\cdots+C_9^9 a^{(9)}b^{(0)}$,以
$(x\sin x)^{(9)}$为例:
\begin{align*}
    (x\sin x)^{(9)} &=x^{(0)}(\sin x)^{(9)}+C_9^1 x^{(1)}(\sin x)^{(8)}+\cdots\quad( x^{(s)}=0,s\geq 2)\\
    &= 9\cos x+9\sin x
\end{align*}
\subsection{递推关系求积分}
\begin{align*}
    \int x^n \cos x d x=\int x^n d \sin x & =x^n \sin x-n \int x^{n-1} \cdot \sin x d x \\
    & =x^n \sin x+n \int x^{n-1} d \cos x \\
    & =x^n \sin x+n x^{n-1} \cos x-n(n-1) \int x^{n-2} \cdot \cos x d x
\end{align*}
则 $I_n=x^n \sin x+n x^{n-1} \cos x-n(n-1) I_{n-2}$ (一般情况我只需要续得这个就行),然后可以根据递推关系式
求解$I_n$。
\newpage
\section{微分方程}
\subsection{齐次N阶线性微分方程(HNOLDE)}
\noindent  基本形式:
\begin{gather*}
    L[y] = y^{(n)} +a_{n-1}y^{(n-1)}+\cdots a_1y'+a_0 =0 \\
    \intertext{代入$y = e^{\lambda x}$得:}
    \lambda^n+a_n\lambda^{n-1}+\cdots +a_1 \lambda+a_0 = 0
\end{gather*}
这个关于$\lambda$得方程被称为特征方程,解这个特征方程得到特征根:
\begin{gather*}
    (\lambda-\lambda_1)^{k_1}(\lambda-\lambda_2)^{k_2}\cdots(\lambda-\lambda_n)^{k_n} = 0
\end{gather*}
其中$\lambda=\lambda_1$被称为$\lambda$得$k_1$重根,其他$lambda$得表示方法类似,那么我们可以得到关于该微分方程解得一组basis,他们可以这样表示:
\begin{align*}
    \{\{r^{k_j-1}e^{\lambda_j}\}_{j=1}^n\} =\{e^{\lambda_j},xe^{\lambda_j},\cdots\}_{j=1}^n
\end{align*}
以二阶线性微分方程为例子:
\begin{align*}
    y''+ay'+by = 0 \\
    \lambda^2+a\lambda+b = 0
\end{align*}
\begin{align*}
     & 1. \Delta>0,\quad \lambda_1 = \frac{-a+\sqrt{a^2-4b}}{2}\quad \lambda_2 =
    \frac{-b-\sqrt{a^2-4b}}{2}                                                       \\
     & \{e^{\lambda_1 t},e^{\lambda_2 t}\}\quad y = Ae^{\lambda_1 t}+e^{\lambda_2 t} \\
     & 2. \Delta = 0, \quad \lambda_1=\lambda_2 = \frac{-a}{2}                       \\
     & \{e^{-at/2},te^{-at/2}\}\quad y = Ae^{-at/2}+Bte^{-at/2}                      \\
     & 3.\Delta<0,\quad \lambda_1 = \frac{-a+i\sqrt{4b-a^2}}{2}\quad \lambda_2
    =\frac{-a-i\sqrt{4b-a^2}}{2}                                                     \\
     & \{e^{i\omega},e^{-i\omega}\}\quad y =e^{-a/2}(Ae^{i\omega}+Be^{-i\omega})
\end{align*}
\subsection{非齐次N阶线性微分方程(IHNOLDE)}
对于非齐次情况,补充以下定理,即可方便得求出特解,齐次通解和非齐次特解相加即为非齐次
方程得通解:HNOLDE右边得0变为$r(x)$就是IHNOLDE。

当$r(x)=e^{\lambda x}S(x)$时,其中$S(x)$为任意多项式,,则非齐次方程有如下特解
$y = e^{\lambda x} q(x)$,$\lambda \neq \lambda_j$时,$q(x)$ 的次数和$S(x)$相同
,$\lambda=\lambda_j$时,$q(x)$的次数为$S(x)$的次数加上$n$,其中n表示n重根。

example:$y''-y=xe^x$
\begin{align*}
     & \lambda_1 = 1,\lambda_2=-1                                       \\
     & y = C_1e^x+C_2e^{-x}+f(x)                                        \\
     & q(x)e^{\lambda x} = (ax^2+bx+c)e^x\quad (\lambda=\lambda_1, n=1) \\
     & y = C_1 e^x+C_2 e^{-x}+\frac{1}{4}(x^2-x)e^x
\end{align*}
\subsection{WKB方法}
当方程$y''+q(x)y=0$中的$q(x)$变化十分缓慢时,可以视为const,然后解这个方程,
这就是WKB方法。
\end{document}
