\documentclass[UTF8]{article}
\usepackage{ctex}
\usepackage{anyfontsize}
\usepackage{amsmath}
\usepackage{bookmark}
\hypersetup{hidelinks}
\usepackage{geometry}
\numberwithin{equation}{section}
\geometry{a4paper,scale=0.8}
% \usepackage{newpxmath}
\usepackage{physics}

\title{固体物理笔记}
\author{lcg}
\begin{document}
\maketitle
\tableofcontents
\newpage
\section{晶体结构}
正交晶系举例:简单立方,体心立方,面心立方,底心正交,如果采用钢球模型来描述的话,简单立方各个顶点的钢球是相切的,
对于体心立方,对角线的钢球是相切的,对于,面心立方,面上的对角线是相切的。
\subsection{晶格的周期性}
晶格的周期性:晶格的周期性可以使用原胞和基矢来描述,原胞指的是晶格
中的最小重复单元,基矢指的是原胞的边矢量。

原胞的体积:$\va{a}_1\dot(\va{a}_2\times\va{a}_3)$

基矢的选取原则:选取一个点作为原点,然后选三个最近邻的点引三条矢量就是
对应的坐标基矢。
简单立方的基矢:$\va{a}_1=a\va{i},\va{a}_2=a\va{j},\va{a}_3=a\va{k}$
,面心立方基矢:$\va{a}_1=\frac{a}{2}(\va{i}+\va{k}),\va{a}_2=
    \frac{a}{2}(\va{k}+\va{j}),\va{a}_3=\frac{a}{2}(\va{i}+\va{j})$,
体心立方基矢:$\va{a}_1=\frac{a}{2}(\va{-i}+\va{-j}+\va{k}),\va{a}_2=\frac{a}{2}(\va{i}-\va{j}+\va{k}),
    \va{a}_3=\frac{a}{2}(\va{i}+\va{j}+\va{k})$。

计算体积直接使用矢量积的行列式形式就可以了:(以体心立方为例子)
\begin{align*}
    V=(\frac{a}{2})^3\left|\begin{array}{ccc}
                               -1 & -1 & 1 \\
                               1  & -1 & 1 \\
                               1  & 1  & 1
                           \end{array}\right|
\end{align*}
复式晶格:复式晶格包含两种或者两种以上的等价原子,复式晶格由他们的子晶格相重叠而成,复式晶格的原胞就是相应子晶格的原胞。


还有另外一种方式来描述晶体的周期性结构:布拉伐格子
\begin{align*}
    \va{R}=\va{r}+l_1\va{a}_1+l_2\va{a}_2+l_3\va{a}_3
\end{align*}
\subsection{晶向,晶面与他们的标志}
晶向:布拉伐格子分布在一系列的直线系上,这些直线系被称为晶列,这些晶列
的方向被称为晶向,如果一个原子到另外一个原子的唯一矢量表示为:$l_1\va{a}_1+l_2\va{a}_2+l_3\va{a}_3$,
则其晶向指数为:$[l_1l_2l_3]$。

我们通过米勒指数来标注晶面,选择一个格点作为原点延三个基矢方向的线段,晶面将其几等分,对应的等分数就是米勒指数,即截距的倒数化为整数:
$(h_1h_2h_3)$

\subsection{倒格子}
倒格子是正格子通过傅里叶变换得到的,变换之前是坐标空间,变换之后是动量空间。倒格子与正格子之间的坐标基矢变换:
\begin{align*}
     & \va{b}_1=2\pi \frac{\va{a}_2\times \va{a}_3}{\va{a}_1\cdot [\va{a}_2\times\va{a}_3]} \\
     & \va{b}_2=2\pi \frac{\va{a}_3\times \va{a}_1}{\va{a}_2\cdot [\va{a}_3\times\va{a}_1]} \\
     & \va{b}_3=2\pi \frac{\va{a}_1\times \va{a}_2}{\va{a}_3\cdot [\va{a}_1\times\va{a}_2]}
\end{align*}
因此可以得到一个非常重要的关系:$\va{b}_i\cdot \va{a}_j=2\pi \delta_{ij}$。
以及衍生出以下性质:
\begin{itemize}
    \item 倒格子的倒格子是原来的正格子
    \item 正格子的原胞体积反比于倒格子的原胞体积
    \item 正格矢与倒格矢满足关系$\va{R}_l\cdot\va{G}_h=2\pi u,u=0,\pm 1,\cdots$
    \item 正格子的一组晶面$(h_1h_2h_3)$与$G_{h_1h_2h_3}$正交
    \item 倒格子的模$|\va{G_{h_1h_2h_3}}|$反比于晶面族$(h_1h_2h_3)$的面间距
\end{itemize}
晶面方程的推导:
\begin{align*}
     & \va{x}=\xi_1\va{a}_1+\xi_2\va{a}_2+\xi_3\va{a}_3                \\
     & \va{G}_{h_1h_2h_3}=h_1\va{b}_1+h_2\va{b}_2+h_3\va{b}_3          \\
     & \va{x}\cdot\va{G}_{h_1h_2h_3}=2\pi (\xi_1h_1+\xi_2h_2+\xi_3h_3)
    =2\pi n
\end{align*}
n取不同整数代表晶面系中不同的晶面。并且由上式可以推导出晶面间距:
\begin{align*}
     & d = \frac{2\pi}{|h_1\vec{b}_1+h_2\vec{b}_2+h_3\vec{b}_3|}
\end{align*}
\subsection{维格纳塞茨原胞和布里渊区}
选取某一点为中心,作其最近邻的所有格点的中垂面,这些平面所包围的
空间就是维格纳塞茨原胞原胞。
倒格子中的维格纳塞茨原胞原被称为第一布里渊区,以此类推,第一布里渊区界面与次近邻与原点的垂直平分面
之间的区域成为第二布里渊区。。。
\subsection{晶体的宏观对称性}
考虑到晶面上的所有布拉伐格点均可表示为:$l_1\va{\alpha}_1+l_2\va{\alpha}_2$.
因此$A'B'=nAB\Rightarrow 1-2\cos \theta = n \in [-1,3]$,因此根据n重轴
的定义,n只能取1,2,3,4,6,n重轴定义:$\frac{2\pi}{n}$。

根据晶体点群,将晶体分为7中晶系和14种布拉伐格子。

在具有立方对称的晶体中,介电常数可以看成标量。\\
证1:
介电常数:
\begin{align*}
    \left(\begin{array}{c}
              D_x \\D_y\\D_z
          \end{array}\right)=
    \left(\begin{array}{ccc}
              \varepsilon_{xx} & \varepsilon_{xy} & \varepsilon_{xz} \\
              \varepsilon_{yx} & \varepsilon_{yy} & \varepsilon_{yz} \\
              \varepsilon_{zx} & \varepsilon_{zy} & \varepsilon_{zz}
          \end{array}
    \right)
    \left(\begin{array}{c}
              E_x \\E_y \\E_z
          \end{array}\right)
\end{align*}
选取z轴为电场方向并以此为旋转轴顺时针旋转$\frac{\pi}{2}$
旋转之前:
\begin{align*}
     & D_x=\varepsilon_{xz}E_z \\
     & D_y=\varepsilon_{yz}E_z \\
     & D_z=\varepsilon_{zz}E_z
\end{align*}
旋转之后:
\begin{align*}
     & D_x'=-D_y \\
     & D_y'=D_x  \\
     & D_z=D_z
\end{align*}
可以推得:$\varepsilon_{xz}=\varepsilon_{yz}=0$,同理绕x,y轴分别
旋转90度,得到:$\varepsilon_{yz}=\varepsilon_{zx}=0,\varepsilon_{xy}=\varepsilon_{zy}=0$.
最后绕[111]方向旋转$\frac{2\pi}{3}$,得到:$\varepsilon_{xx}=\varepsilon_{yy}=\varepsilon_{zz}=\varepsilon_0$。

正交变换:

\newpage
\section{固体结合}
原子的相互作用力:库伦吸引力(长程力)和排斥力(排斥力),其中库伦吸引力
是库伦相互作用,大小正比于$1/r^2$。排斥力是由于泡利不相容原理导致的电子
云之间的强烈排斥作用。当两个力的大小是相等时,处于平衡状态。
\subsection{离子性结合}
以正离子为坐标原点,该正离子和其他所有的离子的库仑作用力的大小为:
\begin{align*}
    F = \sum_{n_1,n_2,n_3}\frac{(-1)^{n_1+n_2+n_3}q^2}{4\pi \varepsilon_0(n_1^2r^2+n_2^2r^2+n_3^2r^2)}
\end{align*}
则一个离子的平均库伦能
\begin{align*}
    U_1 = \frac{1}{2} \sum_{n_1,n_2,n_3}\frac{(-1)^{n_1+n_2+n_3}q^2}{4\pi \varepsilon_0(n_1^2r^2+n_2^2r^2+n_3^2r^2)^{1/2}}
\end{align*}
一个原胞中有两个离子,因此一个原胞的库伦能是$2U_1$
其中:
\begin{align*}
    \alpha =- \sum_{n_1,n_2,n_3}\frac{(-1)^{n_1+n_2+n_3}}{(n_1^2r^2+n_2^2r^2+n_3^2r^2)^{1/2}}
\end{align*}
$\alpha$被称为马德隆常数。

由于电子是费米子满足泡利不相容原理,因此当在两个离子之间也会出现排斥能。这里的
排斥能可以使用势能来表示,两个离子之间的势能为:$\frac{b}{r^n}$,
因此在一个原胞内一对离子的平均排斥能是:$2\times\frac{1}{2}\times 6
    \times \frac{b}{r^n}$。

设晶体有N个原胞,系统的内能为:
\begin{align*}
    U = N(-\frac{A}{r}+\frac{B}{r^n})
\end{align*}
当$\frac{d U}{dr}=0$时,晶体最稳定,此时$r=r_0$为平衡时的近邻距离
,由热力学可以知道$p = -\frac{dU}{dV}$,其中N个原胞的体积:$V=2Nr^3$,
由于一般情况下,晶体一般受到的时大气压$p_0$,大气压对固体的影响很小。
\begin{align*}
    -\frac{dU}{d(2Nr^3)}=p_0\approx 0\Rightarrow \frac{d U}{dr}=0
\end{align*}
当晶体处于大气压数量级的压强下,r取极值$r_0$。此时$U_{r=r_0}=-W_0$。

体变模量定义为:
\begin{align*}
    K=\frac{dp}{-dV/V}=-V\frac{dp}{dV}=V\frac{d(dU/dV)}{dV}=V\frac{d^2U}{dV^2}|_{V=V_0=2Nr_0^3}
\end{align*}

\subsection{共价结合}
因为电子属于全同粒子,满足对称性,构造出波函数的基本形式,然后使用变分法
求出能量本征值。得到成键态上填充两个自旋相反的电子使得能量下降,而反键态能量升高。
未配对的电子在配对时需要使得能量最低而形成的键叫做共价键。

共价键的基本特性:饱和性和方向性。
饱和性:由于共价键只能由未配对的价电子形成,当价电子壳层等于半满或者超过半满时,有8-N个电子是未配对的,方向性:
共价键的强弱决定于两个电子轨道的交叠程度。一个原子在价电子波函数最大
的方向形成共价键。

C原子为什么会形成四个共价键?
发生了$sp^3$轨道杂化,原本C原子的基态电子排布:$1s^22s^23p^2$,发生
轨道杂化电子排布:$1s^22s3p^3$,这样就有四个未配对的电子,可以形成
四个共价键。
\subsection{金属性结合}
金属性结合的基本特点是在结合成晶体时,原来的价电子不在束缚在原子上,转变
在整个晶体内的运动。还有金属性结合是一种体积效应,因为原子越紧凑,库伦
能越低。
\subsection{范德瓦尔斯结合}
范德瓦尔斯结合:两个中性分子之间存在着分子力。
极性分子:正负电荷中心重合,非极性分子:正负电荷中心不重合形成电偶极子

极性分子的结合能:考虑两个平行偶极子的相互作用,两个电偶极子的正电荷之间的
距离为r,第一个电偶极子之间的距离为$l_1$,第二个电偶极子的距离为$l_2$,两个电偶极子之间的作用力是库仑力,
可以得到两个电偶极子之间的库伦能大小为:
\begin{align*}
    U = \frac{1}{4\pi \varepsilon_0}(\frac{q^2}{r}+\frac{q^2}{r+l_2-l_1}-\frac{q^2}{r+l_2}-\frac{q^2}{r-l_1})
\end{align*}
因为$l_2-l_1<<r$,对每一项按照泰勒展开到二阶项:
\begin{align*}
    \frac{1}{1+x}=1-x+x^2
\end{align*}
带入电偶极矩:$p_1=ql_1,p_2=ql_2$,可以得出库伦能为:
\begin{align*}
    u(r)=-2\frac{l_1l_2}{r^2}=-\frac{p_1p_2}{2\pi \varepsilon r^3}
\end{align*}

极性分子与非极性分子的结合:是由于极性分子偶极矩电场的作用下,非极性分子的电子云的中心与核电荷不在重合,导致非极性分子的极化,
产生诱导磁矩。
对于其中的一个极性分子:
\begin{align*}
    E & = \frac{1}{4\pi \varepsilon_0}(\frac{q}{r^2}-\frac{q}{(r-l_1)^2} ) \\
      & = \frac{2p_1}{4\pi \varepsilon  _0 r^3 }
\end{align*}
感生偶极矩为:$p_2=\alpha  E$,则对应的能量为:
\begin{align*}
    U(r)=\frac{-\alpha p_1^2}{2\pi^2 \varepsilon_0^2r^6}
\end{align*}

非极性分子的结合:依靠瞬时偶极矩的相互作用结合而成的,其作用力非常微弱。
第一个原子的瞬时偶极矩$p_1$在r处产生的电场$E\propto p_1/r^3$,第二个原子在电场作用下的感生偶极矩为$p_2=\alpha E$
类比得到库伦能得到两个电偶极子之间的相互作用能为:
\begin{align*}
    U = \frac{p_1p_2}{r^3}=\frac{\alpha p_1^2}{r^6}
\end{align*}
考虑到分子之间还需存在一定的斥力才能达到稳定平衡状态,故相互作用能:
\begin{align*}
    U(r)=-\frac{A}{r^6}+\frac{B}{r^{12}}
\end{align*}
定义勒纳-琼斯势:
\begin{align*}
    U(r)=4\varepsilon[-(\frac{\sigma}{r})^6+(\frac{\sigma}{r})^{12}]
\end{align*}
这个只是两个原子的相互作用能,在整个晶体当中,能量为所有两个原子的
相互作用能求和:
\begin{align*}
    U(r) = \frac{1}{2}\sum_i^{N}\sum_j^{N}4\varepsilon[-(\frac{\sigma}{r_{ij}})^6+(\frac{\sigma}{r_{ij}})^{12}]
\end{align*}
可以得到晶体的总的势能;
\begin{align*}
    U(r) = \frac{N}{2}4\varepsilon[-A_6(\frac{\sigma}{r})^6+A_{12}(\frac{\sigma}{r})^{12}]
\end{align*}
可以根据上式求得晶格常数,结合能以及体变模量。

\newpage
\section{能带理论}
\subsection{布洛赫定理的证明}
首先指出布洛赫定理的内容:当势场具有晶格周期性的时候,波动方程的解具有如下形式:
\begin{align*}
    \psi(r+R_n)=e^{ikR_n}\psi(R)
\end{align*}
引入平移算符$T$,则对任意函数$f(r)$做平移操作:
\begin{align*}
    T_\alpha f(r)=f(r+\alpha_\alpha)
\end{align*}
接下来证明哈密顿量和平移算符是对易的:
\begin{align*}
    \begin{aligned}
        T_{a} H f(\boldsymbol{r}) & =\left[-\frac{h^{2}}{2 m}
            \nabla_{r+a}^{2}+V\left(\boldsymbol{r}+\boldsymbol{\alpha}_{a}\right)\right]
        f\left(\boldsymbol{r}+\boldsymbol{\alpha}_{\alpha}\right)                                  \\
                                  & =\left[-\frac{\hbar^{2}}{2 m} \nabla_{r}^{2}+V(\boldsymbol{r})
            \right] f\left(\boldsymbol{r}+\boldsymbol{\alpha}_{a}\right)
        \\& =H T_{a} f(\boldsymbol{r})
    \end{aligned}
\end{align*}

\subsection{一维势场运动的近自由电子近似}
这类问题可以
\subsection{能态密度和费米面}
能态密度定义:
\begin{align*}
    N(E)=\frac{\Delta N}{\Delta E}=\frac{V}{(2\pi)^3}\iiint dSdq=\frac{V}{(2\pi)^3}\iiint \frac{dS}{\nabla_k E(k)}
\end{align*}
在近自由电子近似当中,由于开始
费米面:所有能量小于$\varepsilon_F$的电子的数目总和为N,可以根据此定义来定义费米速度,动量,能量等物理量。

\end{document}



