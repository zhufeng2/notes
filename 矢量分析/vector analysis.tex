\documentclass[UTF8]{article}
\usepackage{ctex}
\usepackage{anyfontsize}
\usepackage{bookmark}
\hypersetup{hidelinks}
\usepackage{tikz}
\usepackage{pgfplots}
\pgfplotsset{compat=1.5}
\usepackage{tocbibind}
\usepackage{geometry}
\usepackage{amsmath}
\numberwithin{equation}{section}
\usepackage{mathrsfs}
% \usepackage{newpxmath}
\usepackage{physics}
\usepackage{booktabs}
\usepackage{esint} 
\geometry{a4paper,scale=0.8}

\title{vector analysis}
\author{李承高}
\begin{document}
\maketitle
\thispagestyle{empty}
\newpage
\thispagestyle{empty}
\tableofcontents
\newpage

\pagenumbering{arabic}
\section{triple product}
三重积的值是括号里面的两个矢量的线性组合,中间的矢量的系数是正的,例子:
\begin{align*}
    A\times (B\times C) = (A\cdot C)B-(A\cdot B)C
\end{align*}
\section{球坐标}
\begin{align*}
    \begin{cases}
    &x=r\sin\theta \cos \varphi\\
    &y=r\sin\theta \sin \varphi\\
    &z=r\cos \theta
\end{cases}
\end{align*}
\begin{align*}
    f = \sqrt{(\pdv{x}{r})^2+(\pdv{y}{r})^2+(\pdv{z}{r})^2}=1\qquad g=r\qquad h=r\sin\theta
\end{align*}
\begin{align*}
    &\nabla y = \frac{1}{f}\pdv{y}{r}+\frac{1}{g}\pdv{y}{\theta}+\frac{1}{h}\pdv{y}{\varphi}     \\
    &\nabla \cdot \vec{y}= \frac{1}{fgh}(\frac{\partial}{\partial r} (gh\pdv{\vec{y}}{r})
    +\frac{\partial}{\partial \theta}(fh\pdv{\vec{y}}{\theta})+\frac{\partial}{\partial \varphi}(fg\pdv{\vec{y}}{\theta}))\\
    &\nabla^2 y = \frac{1}{fgh}(\frac{\partial}{\partial r} (\frac{gh}{f}\pdv{y}{r})
    +\frac{\partial}{\partial \theta}(\frac{fh}{g}\pdv{y}{\theta})+\frac{\partial}{\partial \varphi}(\frac{fg}{h}\pdv{y}{\theta}))
\end{align*}
\section*{莱布尼兹法则}

\end{document}